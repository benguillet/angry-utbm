\documentclass[a4paper,12pt]{report}

\usepackage{xltxtra} % charge aussi fontspec et xunicode, nécessaires...
\usepackage[frenchb]{babel}
\usepackage[top=2cm, bottom=2cm, left=3cm, right=3cm]{geometry}
\usepackage{url}
\usepackage{hyperref}
\usepackage{amsmath} % oh oui des maths
\usepackage{minted}
\usepackage{eurosym} %utilisation du signe euro
\usepackage{graphicx} %utilisation d'images
\usepackage{fancyhdr} %des en-têtes qui poutrent
\usepackage{titlesec}
\usepackage{upgreek} % lettres grecques
\usepackage{array}
\usepackage[table]{xcolor}
\usepackage[final]{pdfpages}

\usepackage{color} % on en a besoin pour utiliser les couleurs
\definecolor{grey}{rgb}{0.7, 0.7, 0.7}
\definecolor{blue}{rgb}{0.46,0.56,0.68}
\definecolor{red}{rgb}{0.509,0.145,0.114}
\definecolor{orange}{rgb}{1,0.4,0}

\usemintedstyle{tango}

\hypersetup{
     colorlinks=true, %colorise les liens
     breaklinks=true, %permet le retour à la ligne dans les liens trop longs
     urlcolor= blue,  %couleur des hyperliens
     linkcolor= black, %couleur des liens internes
     bookmarksopen=true,            %si les signets Acrobat sont créés,
                                    %les afficher complètement.
     pdftitle={LO43 : Rapport de projet}, %informations apparaissant dans
     pdfauthor={Loïs Aubree, Lucie Boutou, Benjamin GUILLET, Théophile Madet},     %dans les informations du document
     pdfsubject={Rapport de projet LO43},          %sous Acrobat.
     xetex
}

%%%% begin macro allowing marging changed some times :) %%%%
\newenvironment{changemargin}[2]{\begin{list}{}{%
\setlength{\topsep}{0pt}%
\setlength{\leftmargin}{0pt}%
\setlength{\rightmargin}{0pt}%
\setlength{\listparindent}{\parindent}%
\setlength{\itemindent}{\parindent}%
\setlength{\parsep}{0pt plus 1pt}%
\addtolength{\leftmargin}{#1}%
\addtolength{\rightmargin}{#2}%
}\item }{\end{list}}
%%%% end macro %%%%

%\pagestyle{fancyplain}
\pagestyle{fancy}
\fancyhf{}

% Avoir des subsubsection avec des lettres
%\renewcommand{\thesubsubsection}{\thesubsection .\alph{subsubsection}}
\renewcommand{\subsubsectionmark}[1]{\markright{#1}}

\fancyhead[L]{LO43 - A11 - Projet}
\fancyhead[R]{\textsc{Aubree} - \textsc{Boutou}- \textsc{Guillet} - \textsc{Madet}}
\fancyfoot[R]{\thepage}

\renewcommand{\emph}{\textbf}
\newcolumntype{M}[1]{>{\raggedright}m{#1}}

\usepackage{graphicx}
\usepackage{ifthen}
\usepackage{fancybox}
\newcommand{\HRule}{\rule{\linewidth}{1pt}}

\newcommand{\garde}[7]{
% #1 : Auteurs
% #2 : Titre
% #3 : Sous titre
% #4 : Sous sous Titre
% #5 : Date
% #6 : Fichier Image
% #7 : Note bas de page

\thispagestyle{empty}

\begin{changemargin}{-0.5cm}{-0.5cm}

% Auteur + logo
\begin{flushleft}
  \includegraphics[width=5cm]{images/logo_utbm.png}
\end{flushleft}

\begin{flushright}
  \textbf{\large{#1}}
\end{flushright}

\vfill{}

% Titre
\begin{flushright}
  \HRule{}
  \Huge{\textsc{#2}} \\
  \ifthenelse{\equal{#3}{}}
     {}
     {\Large{\textbf{#3}}} \\
  \ifthenelse{\equal{#4}{}}
     {}
    {\large{\textit{#4}}}
  \HRule{}
\end{flushright}

% Date
\ifthenelse{\equal{#5}{}}
  {}
  {\begin{flushright}
     \large{\textbf{#5}}
  \end{flushright}}
\vfill{}

% Logo projet
\ifthenelse{\equal{#6}{}}
  {}
  {\begin{center}
     \includegraphics[width=9cm]{#6}
  \end{center}
  \vfill{}}

% Note de bas de page
\ifthenelse{\equal{#7}{}}
  {}
  {\begin{center}
     \large{#7}
   \end{center}}
\end{changemargin}

\newpage
}


\begin{document}

\garde{Loïs Aubree - Lucie Boutou\\Benjamin Guillet - Théophile Madet} % Authors
{\textbf{Projet de LO43}} % Title
{Réalisation d'un jeu AngryBirds-like} % Sub-title
{Rapport de projet} % Sub-sub-title
{Automne 2011} % Date
{} % Logo
{} % Note bas de page

\tableofcontents
\chapter{Présentation du projet}
Ce semestre en LO43 nous avions la possibilité de créer un jeu ludique similaire à Angry Birds.\\
Le cahier des charges était le suivant :
\begin{itemize}
  \item Environnement : l'environnement du jeu est un terrain ayant des caractéristiques différentes en fonction de la difficulté. Il peux y avoir des cavernes, ponts, …
  \item Personnes : les personnages sont des entités générées aléatoirement de 1 à 10 suivant le niveau de difficulté. Chacune de ces entité ont des propriétés de déplacement différente (vitesse, direction,...). Elles se déplacent de manière aléatoire dans l'environnement en suivant le sol de l'environnement. Leur déplacement est géré par run thread propre.
  \item Oiseaux en colères : Les oiseaux sont de 3 types différentes au moins :
    \begin{itemize}
      \item Type moineaux : ces oiseaux peuvent voler sans réaliser de vol stationnaire, peu longtemps et ont une seule possibilité de ponte d'oeuf de calibre moyen.
      \item Type colibris : ces oiseaux peuvent voler longtemps, même en stationnaire mais n'ont qu'une seul possibilité de ponte de petit calibre.
      \item Type pigeon : ces oiseaux peuvent voler longtemps, en stationnaire et réaliser trois pontes de gros calibre.
    \end{itemize}
\end{itemize}

Le principe du jeu est simple : le joueur a à sa disposition un ensemble d'oiseaux \guillemotleft en colère\guillemotright~,
et doit tuer les ennemis à l'aide d'oeufs pondus en vol. Les oiseaux peuvent également se sacrifier
en réalisant une \guillemotleft attaque suicide\guillemotright~ et tuer les ennemis avec leur propre corps.\\

Le joueur a au départ entre 3 à 5 oiseaux, selon la difficulté du niveau. Chaque type d'oiseaux possède des 
capacités de ponte (nombres d'oeufs) et des possibilités physiques qui lui sont particulières. Le score du joueur est calculé d'après
son nombre d'oiseaux restants à la fin du niveau. Le niveau est fini quand tous les ennemis sont morts.\\

Le jeu est organisé en niveaux de difficultés qui vont de facile à extrême, et dans chaque niveau un nombre défini de niveau.
Seul le niveau 1 de chaque difficulté est au départ disponible, les niveaux 
suivants sont débloqués au fûr et à mesure de la progression du joueur. 
Le joueur ne peut accéder au niveau suivant qu'en ayant validé le précédent.\\
 

\chapter{Organisation et répartition du travail}

\chapter{Spécification}
\section{Diagramme de classes}
\section{Diagramme de séquences}

\chapter{Conception et prise en main}
\section{Créer ou modifier des niveaux}

\chapter{Bilan}
\section{Conclusion}
\section{Améliorations possibles}
Ajout bruits événements
Score sous les niveaux

\end{document}
