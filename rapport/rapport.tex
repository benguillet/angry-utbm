\documentclass[a4paper,12pt]{report}

\usepackage{xltxtra} % charge aussi fontspec et xunicode, nécessaires...
\usepackage[frenchb]{babel}
\usepackage[top=2cm, bottom=2cm, left=3cm, right=3cm]{geometry}
\usepackage{url}
\usepackage{hyperref}
\usepackage{amsmath} % oh oui des maths
\usepackage{minted}
\usepackage{eurosym} %utilisation du signe euro
\usepackage{graphicx} %utilisation d'images
\usepackage{fancyhdr} %des en-têtes qui poutrent
\usepackage{titlesec}
\usepackage{upgreek} % lettres grecques
\usepackage{array}
\usepackage[table]{xcolor}
\usepackage[final]{pdfpages}

\usepackage{color} % on en a besoin pour utiliser les couleurs
\definecolor{grey}{rgb}{0.7, 0.7, 0.7}
\definecolor{blue}{rgb}{0.46,0.56,0.68}
\definecolor{red}{rgb}{0.509,0.145,0.114}
\definecolor{orange}{rgb}{1,0.4,0}

\usemintedstyle{tango}

\hypersetup{
     colorlinks=true, %colorise les liens
     breaklinks=true, %permet le retour à la ligne dans les liens trop longs
     urlcolor= blue,  %couleur des hyperliens
     linkcolor= black, %couleur des liens internes
     bookmarksopen=true,            %si les signets Acrobat sont créés,
                                    %les afficher complètement.
     pdftitle={LO43 : Rapport de projet}, %informations apparaissant dans
     pdfauthor={Loïs Aubree, Lucie Boutou, Benjamin GUILLET, Théophile Madet},     %dans les informations du document
     pdfsubject={Rapport de projet LO43},          %sous Acrobat.
     xetex
}

%%%% begin macro allowing marging changed some times :) %%%%
\newenvironment{changemargin}[2]{\begin{list}{}{%
\setlength{\topsep}{0pt}%
\setlength{\leftmargin}{0pt}%
\setlength{\rightmargin}{0pt}%
\setlength{\listparindent}{\parindent}%
\setlength{\itemindent}{\parindent}%
\setlength{\parsep}{0pt plus 1pt}%
\addtolength{\leftmargin}{#1}%
\addtolength{\rightmargin}{#2}%
}\item }{\end{list}}
%%%% end macro %%%%

%\pagestyle{fancyplain}
\pagestyle{fancy}
\fancyhf{}

% Avoir des subsubsection avec des lettres
%\renewcommand{\thesubsubsection}{\thesubsection .\alph{subsubsection}}
\renewcommand{\subsubsectionmark}[1]{\markright{#1}}

\fancyhead[L]{LO43 - A11 - Projet}
\fancyhead[R]{\textsc{Aubree} - \textsc{Boutou}- \textsc{Guillet} - \textsc{Madet}}
\fancyfoot[R]{\thepage}

\renewcommand{\emph}{\textbf}
\newcolumntype{M}[1]{>{\raggedright}m{#1}}

\usepackage{graphicx}
\usepackage{ifthen}
\usepackage{fancybox}
\newcommand{\HRule}{\rule{\linewidth}{1pt}}

\newcommand{\garde}[7]{
% #1 : Auteurs
% #2 : Titre
% #3 : Sous titre
% #4 : Sous sous Titre
% #5 : Date
% #6 : Fichier Image
% #7 : Note bas de page

\thispagestyle{empty}

\begin{changemargin}{-0.5cm}{-0.5cm}

% Auteur + logo
\begin{flushleft}
  \includegraphics[width=5cm]{images/logo_utbm.png}
\end{flushleft}

\begin{flushright}
  \textbf{\large{#1}}
\end{flushright}

\vfill{}

% Titre
\begin{flushright}
  \HRule{}
  \Huge{\textsc{#2}} \\
  \ifthenelse{\equal{#3}{}}
     {}
     {\Large{\textbf{#3}}} \\
  \ifthenelse{\equal{#4}{}}
     {}
    {\large{\textit{#4}}}
  \HRule{}
\end{flushright}

% Date
\ifthenelse{\equal{#5}{}}
  {}
  {\begin{flushright}
     \large{\textbf{#5}}
  \end{flushright}}
\vfill{}

% Logo projet
\ifthenelse{\equal{#6}{}}
  {}
  {\begin{center}
     \includegraphics[width=9cm]{#6}
  \end{center}
  \vfill{}}

% Note de bas de page
\ifthenelse{\equal{#7}{}}
  {}
  {\begin{center}
     \large{#7}
   \end{center}}
\end{changemargin}

\newpage
}


\begin{document}

\garde{Loïs Aubree - Lucie Boutou\\Benjamin Guillet - Théophile Madet} % Authors
{\textbf{Projet de LO43}} % Title
{Réalisation d'un jeu AngryBirds-like} % Sub-title
{Rapport de projet} % Sub-sub-title
{Automne 2011} % Date
{} % Logo
{} % Note bas de page

\tableofcontents
\chapter{Présentation du projet}
Nous avions ce semestre en LO43 la possibilité de développer un jeu ludique similaire à Angry Birds.\\

Dans notre version, le joueur a à sa disposition un ensemble d'oiseaux \guillemotleft en colère\guillemotright~,
et doit tuer les ennemis à l'aide d'oeufs pondus en vol. Les oiseaux peuvent également se sacrifier comme dans Angry Birds
en réalisant une \guillemotleft attaque suicide\guillemotright~ et tuer les ennemis avec leur propre corps.\\

Le joueur a au départ entre 3 à 5 oiseaux, selon la difficulté du niveau. Chaque type d'oiseaux possède des 
capacités de ponte (nombres d'oeufs) et des possibilités physiques qui lui sont particulières. Le score du joueur est calculé d'après
son nombre d'oiseaux restants à la fin du niveau. Celui-ci est fini quand tous les ennemis sont morts.\\

Le jeu est organisé en suivant des difficultés qui vont de facile à extrême, et 
chacune d'elle possède un nombre défini de niveaux.
Seul le niveau 1 de chaque difficulté est au départ disponible, les niveaux 
suivants sont débloqués au fur et à mesure de la progression du joueur. 
Ce dernier ne peut accéder au niveau suivant qu'en ayant validé le précédent.\\
 

\chapter{Organisation et répartition du travail}
Notre groupe était composé de quatre personnes, nous avons régulièrement organisé 
des réunions chez les uns et les autres pour avancer le projet et faire le point.\\

La partie spécification UML a été faîte une première fois tous ensemble afin de se
mettre d'accord sur une base de départ puis a évolué pour mieux coller au cahier des
charges : respect du pattern MVC, utilisation des threads et du polymorphisme etc.\\

Nous avons utilisé Eclipse, Skype et un dépot GitHub pour la synchronisation et la gestion des versions.
GitHub possède également un suivi des fonctionnalités à ajouter ou à corriger. Ce qui a 
permit à chacun de travailler les classes et fonctionnalités nécessaires qui lui plaisaient.\\

Globalement on peut dire que le travail s'est réparti ainsi :
\begin{itemize}
 \item Benjamin  : base du projet (respect du pattern MVC), gestion du joueur ;
 \item Loïs : collisions et envoi des oiseaux ;
 \item Lucie: décors et menus ;
 \item Théophile : gestion des sauvegardes, collisions.
\end{itemize}
~\\
Mais tout le monde a par moment débordé sur le \guillemotleft secteur\guillemotright~ 
des autres pour corriger un bug ou améliorer une fonctionnalité. Rien n'était figé.

\chapter{Spécification}
\section{Diagramme de classes}
\section{Diagramme de séquences}

\chapter{Conception et prise en main}
\section{Créer ou modifier des niveaux}

\chapter{Bilan}
\section{Conclusion}
Ce projet était très intéressant et motivant. Son côté ludique nous a permis
d'apprendre le java avec une vraie motivation et une bonne ambiance. Un groupe 
important comme le notre nécessite une bonne communication afin que chacun sache 
le travail à accomplir et éviter les répétitions. Heureusement aujourd'hui nous avons de nombreux outils 
à notre disposition pour nous faciliter la tâche, en plus des classiques réunions ou e-mails.

\section{Améliorations possibles}
Un projet comme celui-ci est toujours perfectible, voici quelques-unes des améliorations 
qui nous viennent en tête et que nous aurions bien voulu implémenter :\\
\begin{itemize}
  \item Il serait facile d'ajouter une musique d'ambiance, des bruits lors des 
  collisions ou lorsque l'on gagne ou perds ;
  \item On pourrait afficher le meilleur score du joueur actuel sous chaque niveaux ;
  \item Les ennemis pourraient gérer la gravité, c'est à dire s'ils sont placés 
      sur un bloc en hauteur, descendre jusqu'en bas lors de la destruction de ce bloc.
\end{itemize}
\end{document}
